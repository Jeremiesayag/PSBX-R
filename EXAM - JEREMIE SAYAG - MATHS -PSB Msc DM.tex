{\rtf1\ansi\ansicpg1252\cocoartf2577
\cocoatextscaling0\cocoaplatform0{\fonttbl\f0\fswiss\fcharset0 Helvetica;}
{\colortbl;\red255\green255\blue255;}
{\*\expandedcolortbl;;}
\paperw11900\paperh16840\margl1440\margr1440\vieww11520\viewh8400\viewkind0
\pard\tx566\tx1133\tx1700\tx2267\tx2834\tx3401\tx3968\tx4535\tx5102\tx5669\tx6236\tx6803\pardirnatural\partightenfactor0

\f0\fs24 \cf0 \\documentclass\{article\}\
\\usepackage[utf8]\{inputenc\}\
\\usepackage\{graphicx\}\
\\graphicspath\{/Users/jeremiesayag/Desktop/MASTER SPE DM/Programmation R (Statistiques et data mining 1) - M. LAUDE/PSBX-R/\}\
\
\\title\{EXAM - Maths - PSB Msc DM\}\
\\author\{JEREMIE SAYAG \}\
\\date\{January 2021\}\
\
\\begin\{document\}\
\
\\maketitle\
Criteres : Qualite de la presentation, pedagogie, decouverte, utilisation, simplicit\'e9\
\\section\{Analysis of a PhD thesis\} \
\\paragraph\{Soukaina EL GHALDY\}\
\\subparagraph\{https://github.com/soukainaElGhaldy/PSB-X/blob/main/Mathematics/maths.pdf\}\
\\subsection\{What's inside\}\
L'auteur nous presente une analyse de la these de doctorat de Baptiste Barreau et en particulier son travail sur un algorithme de Machine Learning appele "Expert Network".\
Il s'agit de permettre une prediction des interets futurs de clients sur les marches financiers.\
\
La problematique tres interessante a laquelle s'attaque Baptiste Barreau est:\
"\'c0 une date t, quel investisseur est int\'e9ress\'e9 par l\'92achat/vente de quel actif financier ?"\
\
\\subsection\{Focus\}\
Concentrons nous sur la formule suivante :\
\
\\includegraphics\{phdmathfinance-obligation.png\}\
\
Cette equation permet de calculer la valeur d'une obligation a un certain nombre d'annees ecoules n, \'e0 p\'e9riodicit\'e9 p, \'e0 un prix d\'92\'e9mission Pe et un taux d\'92int\'e9r\'eat annuel Ir.\
\
L'auteur s'est rendu compte qu'au fur et a mesure des annees la courbe des taux d'interet de l'obligation aura une forme exponentielle.\
Cette periodicite exponentielle est appele "Continuous Compounding".\
\\subsection\{V Criterias\}\
\\begin\{itemize\} \
   \\item Qualit\'e9 de la presentation: Bonne presentation, utilisation de schema et graphique.\
   \\item Pedagogie: L'auteur tente de nous faire comprendre un sujet complique pour les novices en finances. Peut etre, l'utilisation de plus d'exemples concret nous aurait aide.\
   \\item Decouverte: Aimant la finance, la decouverte totale de cette these m'a particulierement plu.\
   \\item Utilisation: Cela est interessant mais particulierement utile surtout pour les personnes travaillant dans le monde de la fincance.\
   \\item Simplicit\'e9: Dure a comprendre pour les novices en finances, il faut aller chercher des explications plus poussees pour mieux comprendre.\
\\end\{itemize\} \
\\subsection\{The End\}\
C'est un sujet qui m'interesse particulierement,et avec l'essort des fintech la data et la finance font un bon mix.\
\
---------------------------------------------------------------------------\
\\section\{Algorithme Genetique\}\
\\paragraph\{COMLAN Florine & HOUNTONDJI Ramya\}\
\\subparagraph\{https://github.com/fcom-stack/PSBX/blob/main/Algorithme_genetique/Algo_genetique.pdf\}\
\\subsection\{What's inside\}\
Une introduction et explication d'un type d'algorithme particulier que sont les algorithmes genetiques.\
\
Ces algorithmes ont la particularite d'avoir un mode de fonctionnnement similaire au corps humain.\
Les auteurs nous detaillent dans un sujet tres complet le fonctionnement de ces algorithmes genetiques.\
\
\\subsection\{Focus\}\
Expliquons une fonction qu'on appelle "Fonction de Fitness" :\
\
\\includegraphics\{fonctiondefitness.png\}\
\
La fonction de fitness est une fonction qui permet de mesurer la performance d'une solution candidate par rapport au probleme propose.\
\
C'est une fontion de performance qui prend en valeur d'entree une solution candidate et dont le calcul de la valeur de performance est effectu\'e9 de mani\'e8re r\'e9p\'e9t\'e9e dans un algorithme genetique et qui doit \'eatre\
suffisamment rapide.\
\
\\subsection\{V Criterias\}\
\\begin\{itemize\} \
   \\item Qualit\'e9 de la presentation: Presentation tres complete qui permet de cerner les enjeux des algorithmes genetiques;\
   \\item Pedagogie: Utilisation de beaucoup de schemas d'explications qui nous permettent de comprendre;\
   \\item Decouverte: Je ne connaissais pas du tout l'existence de ce genre d'algorithme, tres interressant ! ;\
   \\item Utilisation: Il est necessaire d'avoir plusieurs outils a porte de mains pour utiliser le plus approprie en fonction de notre problematique;\
   \\item Simplicit\'e9: Essayez de se rapprocher du fonctionnement de cette formidable machine qu'est le corps humain est difficile mais l'auteur tente de nous faire comprendre ce type d'algorithme de maniere tres didactique;\
\\end\{itemize\} \
\\subsection\{The End\}\
Presentation tres complete des algorithmes genetiques et leur fonctionnement.\
--------------------------------\
\\section\{EPARS Early Prediction of At-risk Students with\
Online and Offline Learning Behaviors\}\
\\paragraph\{AUFRERE Thuy, MAZZUCATO Claire, REN Claude\}\
\\subparagraph\{https://github.com/clairemazzucato/PSBX\}\
\\subsection\{What's inside\}\
Les auteurs nous proposent une reflexion interessante sur un papier de recherche qui porte sur la pr\'e9diction du comportement d\'92apprentissage des \'e9tudiants \'e0 risque (STAR) afin d\'92intervenir \'e0 temps en cas d\'92abandon scolaire.\
\
2 obesrvations ont ete faites : \
\
1- Les \'e9tudiants \'e0 risque ne disposent pas d\'92une routine d\'92\'e9tude r\'e9guli\'e8re et claire.\
\
2- L'environement des etudiants a risque sont en general a risque eux memes\
\\subsection\{Focus\}\
Considerons l'equation suivante :\
\
\\includegraphics\{Papier EPARS.png\}\
\
Le fait d\'92avoir parcouru le voisinage des noeuds permettent donner\
un ensemble qui est appel\'e9 Ns(u). Les fonctions f(.) sont des fonctions de repr\'e9sentation des noeuds en question.\
\\subsection\{V Criterias\}\
\\begin\{itemize\} \
   \\item Qualit\'e9 de la presentation: Presentation succinte\
   \\item Pedagogie: Pas assez d'exemple d'utilisation\
   \\item Decouverte : Totalement nouveau pour moi \
   \\item Utilisation: utilisation par des personnes specifiques (enseignants, chercheurs, etc.)\
   \\item Simplicit\'e9: Complique a utiliser, notions mathematiques complexes.\
\\end\{itemize\} \
\\subsection\{The End\}\
Malgre la difficulte de comprehension que j'ai pu avoir j'ai adore decouvrir ce papier de recherche tres interessant.\
\
--------------------------------\
\\section\{4\}\
\\paragraph\{ABBES Ahmed, BEN YOUSSEF Salah, BENSALEM Akram\}\
\\subparagraph\{https://github.com/Ahmed-Abbes5/psbx\}\
\\subsection\{What's inside\}\
Une introduction aux algorithmes d'apprentissage automatique qui est le fondement du machine learning et de l'intelligence artificielle.\
\
Ces algorithmes se fonde sur des approches math\'e9matiques\
et statistiques pour donner aux ordinateurs la capacit\'e9 d\'92\'ab apprendre \'bb \'e0 partir de donn\'e9es, c\'92est-\'e0-dire d\'92am\'e9liorer leurs performances \'e0 r\'e9soudre des t\'e2ches sans \'eatre explicitement programm\'e9s pour chacune. \
\
\\subsection\{Focus\}\
\
Focalisons nous sur les algorithmes de regression :\
\
La r\'e9gression concerne la mod\'e9lisation de la relation entre les variables qui est affin\'e9e de mani\'e8re it\'e9rative \'e0 l\'92aide d\'92une mesure d\'92erreur dans les pr\'e9dictions faites par le mod\'e8le.\
\
\
Les m\'e9thodes de r\'e9gression sont une b\'eate de somme des statistiques et ont \'e9t\'e9 coopt\'e9es dans l\'92apprentissage automatique statistique. Cela peut pr\'eater \'e0 confusion car nous pouvons utiliser la r\'e9gression pour d\'e9signer la classe du probl\'e8me et la classe de l\'92algorithme.\
\
\\subsection\{V Criterias\}\
\\begin\{itemize\} \
   \\item Qualit\'e9 de la presentation: Bonne presentation mais manque d'exemple\
   \\item Pedagogie: Le manque d'exemple nous fait defaut pour parfaitement comprendre les concepts expliques ici.\
   \\item Decouverte: Nous connaissions les principes de ces algorithmes mais le travail des auteurs nous a permis de mieux comprendre ces concepts.\
   \\item Utilisation: Ces algorithmes sont a la base du machine learning et de l'intelligence artificielle et donc necessaire.\
   \\item Simplicit\'e9: Plutot simple de prime abord mais plus complique lorsque l'on doit les utiliser.\
\\end\{itemize\} \
\\subsection\{The End\}\
Nous remercions les auteurs de nous avoir permis de nous replonger dans les algorithmes de base du machine learning et de l'I.A qui nous a permis de tout reviser.\
\
--------------------------------\
\\section\{Na\'efve Bayes Classifier\}\
\\paragraph\{Jordy HOUNSINOU\}\
\\subparagraph\{https://github.com/Jordyhsn/PSB_Hounsinou\}\
\\subsection\{What's inside\}\
On profite tous meme sans s'en rendre compte d'un algorithme d'apprentissage automatique qui se base sur la classification de "Na\'efve Bayes Classifier" et sur le theoreme de Bayes.\
\
En effet, il s'agit du fameux filtre anti-spam present aujourd'hui dans tout nos comptes emails.\
\\subsection\{Focus\}\
Regardons la fameuse loi de Bayes:\
\
\\includegraphics\{loi de bayes.png\}\
\
Il s'agit d'une loi statistique, qui permet de connaitre la probabilite que l'evenement A se realise en fonction de l'evenement B.\
\
Le th\'e9or\'e8me de Bayes vise \'e0 calculer les probabilit\'e9s a posteriori d\'92un \'e9v\'e9nement en fonction des probabilit\'e9s a priori de cet \'e9v\'e9nement.\
A priori et a posteriori s\'92entendent par rapport \'e0 la connaissance d\'92une information. L\'92exemple typique est celui du diagnostic : a priori on juge que le patient a une telle probabilit\'e9 d\'92avoir la maladie M.\
\
Cette loi permet de repondre a cette question:\
\
Que devient a posteriori cette probabilit\'e9 lorsque\
l\'92on apprend le r\'e9sultat de tel examen clinique ou\
biologique ?\
\
\\subsection\{V Criterias\}\
\\begin\{itemize\} \
   \\item Qualit\'e9 de la presentation: Presentation tres complete\
   \\item Pedagogie: Utilisation d'exemple concret par l'auteur, tres didactique\
   \\item Decouverte: Nous connaissions ce theoreme de Bayes depuis le lycee, il est interessant de le revoir.\
   \\item Utilisation: Tres utilise en statistique\
   \\item Simplicit\'e9: Assez simple a utiliser\
\\end\{itemize\} \
\\subsection\{The End\}\
L'auteur nous permet de nous replonger dans nos souvenirs de lycee en nous presentant la fameuse loi de Bayes.\
\
--------------------------------\
\\section\{Les reseaux de neurones antagonistes generatifs\}\
\\paragraph\{Jeremie SAYAG\}\
\\subparagraph\{https://github.com/Jeremiesayag/PSBX-R\}\
\\subsection\{What's inside\}\
Une introduction aux GAN "Generative Adversarial Nets" ou reseaux de neurones antagonistes generatifs en fran\'e7ais, qui permet entre autre la realisation de ce site :\\href\{thispersondoesnotexist.com\}.\
\
Ce reseau de neurones permet de de construire des modeles d'intelligence articielle capable de generer un contenu articiel realiste.\
\\subsection\{Focus\}\
Le principe sous jacent de ces neurones est le suivant :\
\
Considerons 2 entites : Un generateur et un discriminateur.\
\
Le generateur a comme role de creer un contenu realiste\
\
Le discrimateur a comme role de verifier si un contenu est reel ou artificiel.\
\
Les GAN ont commme objectifs de creer un generateur tellement efficace qu'il cree une image fausse en faisant penser au discriminateur que l'image est vraie.\
\
\\subsection\{V Criterias\}\
\\begin\{itemize\} \
   \\item Qualit\'e9 de la presentation: Presentation plutot complete mais a completer avec des exemples concret.\
   \\item Pedagogie: Manque de visuel qui pourrait permettre une meilleure pedagogie\
   \\item Decouverte: Nous avions adore le concept du site internet, il a donc ete interessant de comprendre comment cela etait possible.\
   \\item Utilisation: Utilisation de plus en plus courante dans le futur avec l'essor des deep fake.\
   \\item Simplicit\'e9: Concept mathematiques plutot complique au debut mais apres quelques heures de recherche cela devient un peu plus simple.\
\\end\{itemize\} \
\\subsection\{The End\}\
Nous avons essaye de trouver un sujet interessant, et de l'expliquer de la maniere la plus simple possible.\
Les GAN sont un sujet qui nous interesse particulierement a la lumiere de toutes les fakes news qui sont autour de nous dans le context actuel.\
\
\
\\section\{The Real End\}\
Apres de longues heures de recherches mes camarades et moi meme avons essaye de presenter un sujet qui nous interesse et interesse les autres.\
Certains de mes camarades ont utilise beaucoup de schema et d'image ainsi que d'exemple concret qui je le pense est un reel avantage pour une bonne comprehension.\
\
En effet, un sujet theorique aussi facile qu'il soit, pour bien le comprendre rien de mieux qu'une mise en pratique. \
Cela est encore plus vrai avec ces sujets difficiles.\
\
Concernant, le choix des sujets mon coup de coeur est pour le sujet sur l'Expert Network car cela concerne la finance, qui m'interesse.\
\
Nous aurions pu avoir une meilleure pedagogie d'ensemble en creeant des diaporamas ou videos explicatives de type tuto nous permettant de capter mieux notre auditoire et permettre une meilleure comprehension.\
\\end\{document\}\
}